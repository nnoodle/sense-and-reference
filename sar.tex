%%% Sense and Reference: by Gottlob Frege, translated by Max Black.
%%% From The Philosophical Review, Volume 57, Issue 3 (May, 1948),
%%% pg. 209-230.

%%% Both the original and the translation are in the public domain.

\documentclass[twoside,12pt,a4paper]{article}
\usepackage[T1]{fontenc}
\usepackage{baskervald}
\usepackage{fancyhdr}
\usepackage{enumitem}
\usepackage{type1cm}
\usepackage{lettrine}
\usepackage{marginnote}

\pagestyle{fancy}
\fancyhf{}
\renewcommand{\headrulewidth}{0pt}
\fancyhead[CE]{\it THE PHILOSOPHICAL REVIEW}
\fancyhead[CO]{FREGE'S {\it UEBER SINN UND BEDEUTUNG}}
\fancyfoot[C]{\ifnum\value{page}<4\relax\else\thepage\fi}

\newcommand{\footnoteAlph}[2][\thefootnote]{%
  \renewcommand{\thefootnote}{\Alph{footnote}}%
  \footnote[#1]{#2}%
  \renewcommand{\thefootnote}{\arabic{footnote}}}

\setlength{\LettrineDepth}{10pt}

\begin{document}

\thispagestyle{empty}
\begin{center}
  {\LARGE A TRANSLATION OF FREGE'S
    \vskip 0.5em
    {\it UEBER SINN UND BEDEUTUNG}}
  \vskip 1.5em
  %% {\large
  %%   \lineskip .5em
  %%   \begin{tabular}[t]{c}
  %%     Max Black
  %%   \end{tabular}\par}
  %% \vskip 1em
\end{center}

\begin{center}
  \subsection*{\it Introductory Note}
\end{center}

\lettrine[loversize=0.2]{\textbf{T}}{\ \,HE LOGICAL} and mathematical
writings of Gottlob Frege are gradually emerging from undeserved
neglect. After a lapse of half a century, it is becoming generally
recognized that his views on topics of fundamental importance in the
foundations of mathematics and the philosophy of symbolism deserve
very serious attention. And many feel that were Frege's judgment
differs from the more generally received doctrine he is more likely to
be right than his opponents. This process of revaluation would be
hastened if Frege's writings were more generally accessible.

It has therefore seemed worthwhile to translate his famous paper on
``Sinn und Bedeutung,'' published in 1892 in the {\it Zeitschrift
  f\"ur Philosophie und philosophische Kritik} (vol. 100, pp. 25-50).

The point of central interest is Frege's distinction between sense
({\it Sinn}) and designation or denotation or, as I have chosen to
call it, reference ({\it Bedeutung}). This corresponds, in some
measure, to a distinction which other philosophers have made between
\emph{connotation} and \emph{denotation}, or \emph{intension} and
\emph{extension} (or even \emph{description} and \emph{acquaintance}).
But Frege's distinction is not to be identifies with any of these. I
doubt that many philosophers would accept Frege's demonstration that
the referent of a declarative sentence is either \emph{the true} or
\emph{the false}; or that many will sympathize with his dogged
persistence in examining the relevance of his distinction to the
manifold types of subordinate clauses which may be exemplified in
complex sentences. But if his doctrines occasionally seem implausible,
they always deserve detailed refutation; and the effort of the
following his argument is rewarded by flashes of penetrating insight.

In trying to prepare a \emph{literal} translation which will not sound
foreign, one runs into obvious difficulties. In the present instance,
these are aggravated by the novelty of Frege's ideas and the
consequent lack of a settled terminology of their expression. (Thus
some may object to my choice of ``refer to'' for {\it bedeuten} and
``referent'' for {\it Bedeutung}. But ``denote'' is misleading,
``designatum'' is clumsy, and ``nominatum''---Carnap's suggestion---too
new for general acceptance.) The translator is also harassed by
Frege's fondness for parenthetical qualifications and his liberal use
of untranslatable German particles. I have tried to be faithful to
Frege's intentions, even at the cost of occasional clumsiness.

The following is a list of the chief terms used by Frege in
\emph{technical senses} (often diverging from common meanings of the
words) together with the English equivalents chosen.

\vskip 0.5em
\begin{tabular}{p{12em}@{\hspace{3em}}p{15em}} %[bct]{lcrpmb|}
  Anschauung & Experience\\
  Art des Gegebenseins & Mode of presentation\\
  Bedeutung & Reference [for the process]\\
             & Referent [for the object]\\
  Begriff & Concept or predicate\\
  Begriffsausdruck & Predicate expression\\
  Behauptungssatz & Declarative sentence\\
  Bild & Representation\\
  Eigenname & Proper name\\
  Erkenntniswerth & Cognitive value\\
  Gedanke & Thought [a sense peculiar to F., for which proposition has
            also been suggested]\\
  Gedankentheil & Part of a thought\\
  Gew\"ohnliche Bedeutung & Customary reference\\
  Hauptsatz & Main clause\\
  Nebensatz & Subordinate clause\\
  Sinn & Sense\\
  Satz & Sentence or clause [according to context]\\
  Unbestimmt andeutender
  Bestandtheil & Indefinite indicator
                 [corresponds approximately to
                 ``variable'' in modern
                 usage]\\
  Vorstellung & Conception\\
  Warheitswerth & Truth value\\
\end{tabular}
\vskip 0.5em

The footnotes are Frege's own, except for five interpolated comments,
shown by the use of small capital letters as markers. Marginal numbers
indicate original pagination.

\lfoot{\it Cornell University}
\rfoot{MAX BLACK}

\newpage
\lfoot{}
\rfoot{}
\title{\vspace{-2em}ON SENSE AND REFERENCE}
\author{Gottlob Frege}
\date{\vspace{-2em}}
\maketitle

\marginnote{25}
Identity\footnote[1]{I use this word strictly and understand ``$a=a$''
  to have the sense of ``a is the same as b'' or ``a and b
  coincide.''} gives rise to challenging questions which are not
altogether easy to answer. Is it a relation? A relation between
objects, or between names or signs of objects? In my {\it
  Begriffsschrift}\footnoteAlph[1]{The reference is to Frege's {\it
    Beigriffsschrift, eine der arithmetischen nachgebildete
    Farmelsprache des reinen Denkens} (Halle, 1879).} I assumed the
latter. The reasons which seem to favor this are the following: $a=a$
and $a=b$ are obviously statements of differing cognitive value; $a=a$
holds {\it a priori} and, according to Kant, is to be labeled
analytic, while statements of the form $a=b$ often contain very
valuable extensions of our knowledge and cannot always be established
{\it a priori}. The discovery that the rising sun is not new every
morning, but always the same, was of very great consequence to
astronomy. Even today the identification of a small planet or a comet
is not always a matter of course. \marginnote{26}
Now if we were to regard identity as a relation between that which the
names ``$a$'' and ``$b$'' designate, it would seem that $a=b$ could
not differ from $a=a$ (i.e. provided $a=b$ is true). A relation would
thereby be expressed of a thing to itself, and indeed one in which
each thing stands to itself but to no other thing. What is intended to
be said by $a=b$ seems to be that the signs or names ``$a$'' and
``$b$'' designate the same thing, so that those signs themselves would
be under discussion; a relation between them would be asserted. But
this relation would hold between the names or signs only insofar as
they named or designated something. It would be mediated by the
connection of each of the two signs with the same designated thing.
But his is arbitrary. Nobody can be forbidden to use any arbitrarily
producible event or object as a sign for something. In that case the
sentence $a=b$ would no longer refer to the subject matter, but only
to its mode of designation; we would express no proper knowledge by
its means. But in many cases this is just what we want to do. If the
sign ``$a$'' is distinguished from the sign ``$b$'' only as object
(here, by means of its shape), not as a sign (i.e., not by the manner
in which it designates something), the cognitive value of $a=a$
becomes essentially equal to that of $a=b$, provided $a=b$ is true. A
difference can arise only if the difference between the signs
corresponds to a difference in the mode of presentation of that which
is designated. Let $a, b, c$ be the lines connecting the vertices of a
triangle with the midpoints of the opposite sides. The point of
intersection of $a$ and $b$ is then the same as the point of
intersection of $b$ and $c$. So we have different designations for the
same point, and these names (``Point of intersection of $a$ and $b$,''
``Point of intersection of $b$ and $c$'') likewise indicate the mode
of presentation; and hence the statement contains true knowledge.

It is natural, now, to think of there being connected with a sign
(name, combination of words, letter), besides that to which the sign
refers, which may be called the referent of the sign, also what I
would like to call the \emph{sense} of the sign, wherein the mode of
presentation is contained. In our example, accordingly, the referents
of the expressions \marginnote{27} ``the point of intersection of $a$
and $b$'' and ``the point of intersection of $b$ and $c$'' would be
the same as that of ``morning star,'' but not the sense.

It is clear from the context that by ``sign'' and ``name'' I have here
understood any designation representing a proper name, whose referent
is thus a definite object (this word taken in the widest range), but
no concept and no relation, which shall be discussed further in
another article\footnoteAlph[2]{See his ``Ueber Begriff und
  Gegenstand'' in {\it Vierteljahrsschrift f\"ur wissenschaftliche
    Philosophie} (XVI [1892], 192--205)}. The designation of a single
object can also consist of several words or other signs. For brevity,
let every such designation be called a proper name.

The sense of a proper name is grasped by everybody who is sufficiently
familiar with the language or totality of designations to which it
belongs;\footnote[2]{In the case of an actual proper name such as
  ``Aristotle'' opinions as to the sense may differ. It might, for
  instance, be taken to be the following: the pupil of Plato and
  teacher of Alexander the Great. Anybody who does this will attach
  another sense to the sentence ``Aristotle was born in Stagira'' than
  will a man who takes as the sense of the name: the teacher of
  Alexander the Great who was born in Stagira. So long as the referent
  remains the same, such variations of sense may be tolerated,
  although they are to be avoided in the theoretical structure of a
  demonstrative science and ought not to occur in a complete
  language.} but this serves to illuminate only a single aspect of the
referent, supposing it to exist. Comprehensive knowledge of the
referent would require us to be ale to say immediately whether every
given sense belongs to it. To such knowledge we never attain.

The regular connection between a sign, its sense, and its referent is
of such a kind that to the sign there corresponds a definite sense and
to that in turn a definite referent, while to a given referent (an
object) there does not belong only a single sign. The same sense has
different expressions in different languages or even in the same
language. To be sure, exceptions to this regular behavior occur. To
ever expression belonging to a complete totality of signs, there
should certainly correspond a definite sense; but natural languages
often do \marginnote{28} not satisfy this condition, and one must be
content if the same word has the same sense in the same context. It
may perhaps be granted that every grammatically well-formed expression
representing a proper name always has a sense. But this is not to say
that to the sense there also corresponds a referent. The words ``the
celestial body most distant from the earth'' have a sense, but it is
very doubtful if they also have a referent. The expression ``the least
rapidly convergent series'' has a sense; but it is known to have no
referent, since for every given convergent series, another convergent,
but less rapidly convergent, series can be found. In grasping a sense,
one is not certainly assured of a referent.

If words are used in the ordinary way, one intends to speak of their
referents. It can also happen, however, that one wishes to talk about
the words themselves or their sense. This happens, for instance, about
the words themselves or their sense. One's own words then first
designate words of the other speaker, and only the latter have their
usual referents. We then have signs of signs. In writing, the words
are in this case enclosed in quotation marks. Accordingly, a word
standing between quotation marks must not be taken as having its
ordinary referent.

In order to speak of the sense of an expression ``$A$'' one may simply
use the phrase ``the sense of the expression `$A$.'\,'' In reported
speech one talks about the sense---e.g., of another person's remarks. It
is quite clear that in this way of speaking words do not have their
customary referents but designate what is usually their sense. In
order to have a short expression, we will say: In reported speech,
words are used \emph{indirectly} or have their \emph{indirect}
referents. We distinguish accordingly the \emph{customary} from the
\emph{indirect} referent of a word; and its \emph{customary} sense
from its indirect sense. The indirect referent of a word is
accordingly its customary sense. Such exceptions must always be borne
in mind if the mode of connection between sign, sense, and referent in
particular cases is to be correctly understood.

\marginnote{29}
The referent and sense of a sign are to be distinguished from the
associated conception. If the referent of a sign is an object
perceivable by the senses, my conception of it is an internal
image,\footnote[3]{We can conclude with the conceptions the direct
  experiences in which sense-impressions and activities themselves
  take the place of the traces which they have left in the mind. The
  distinction is unimportant for our purpose, especially since
  memories of sense-impressions and activities always help to complete
  the conceptual image. One can also understand direct experience as
  including any object, in so far as it sensibly perceptible or
  spatial.} arising from memories of sense impressions which I have
had and activities, both internal and external, which I have
performed. Such a conception is often saturated with feeling; the
clarity of its separate parts varies and oscillates. The same sense is
not always connected, even in the same man, with the same conception.
The conception is subjective: One man's conception is not that of
another. There result, as a matter of course, a variety of differences
in the conceptions associated with the same sense. A painter, a
horseman, and a zoologist will probably connect different conceptions
with the name ``Bucephalus.'' This constitutes an essential
distinction between the conception and the sign's sense, which may be
the common property of many and therefore is not a part or a mode of
the individual mind. For one can hardly deny that mankind has a common
store of thoughts which is transmitted from one generation to
another.\footnote[4]{Hence it is inadvisable to use the word
  ``conception'' to designate something so basically different.}

In the light of this, one need have no scruples in speaking simply of
\emph{the} sense, whereas in the case of a conception one must
precisely indicate to whom it belongs and at what time. It might
perhaps be said: Just as one man connects this conception and another
that conception with the same word, so also one man can associate this
sense and another that sense. But there still remains a difference in
the mode of connection. They are not prevented from grasping the same
sense; \marginnote{30} but they cannot have the same conception. {\it
  Si duo idem faciunt, non est idem}.

If two persons conceive the same, each still has his own conception.
It is indeed sometimes possible to establish differences in the
conceptions, or even in the sensations, of different men; but an exact
comparison is not possible, because we cannot have both conceptions
together in the same consciousness.

The referent of a proper name is the object itself which we designate
by its means; the conception, which we thereby have, is wholly
subjective; in between lies the sense, which is indeed no longer
subjective like the conception, but is yet not the object itself. The
following analogy will perhaps clarify these relationships. Somebody
observes the moon through a telescope. I compare the moon itself to
the referent; it is the object of the observation, mediated by the
real image projected by the object glass in the interior of the
telescope, and by the retinal image of the observer. The former I
compare to the sense, the latter to the conception or experience. The
optical image in the telescope is indeed one-sided and depended upon
the standing point of observation; but it is still objective, inasmuch
as it can be used by several observers. At any rate it could be
arranged for several to use it simultaneously. But each one would have
his own retinal image. On account of the diverse shapes of the
observers' eyes, even a geometrical congruence could hardly be
achieved, and a true coincidence would be out of the question. This
analogy might be developed still further, by assuming $A$'s retinal
image made visible to $B$; or $A$ might also see his own retinal image
in a mirror. In this way we might perhaps show how a conception can
itself be taken as an object, but as such is not for the observer what
it directly is for the person having the conception. But to pursue
this would take us too far afield.

We can now recognize three levels of difference between words,
expressions, or whole sentences. The difference may concern at most
the conceptions, or the senses but not the referent, or finally, the
\marginnote{31}
referent as well. With respect to the first level, it is to be noted
that, on account of the uncertain connection of conceptions with
words, a difference may hold for one person, which another does not
find. The difference between a translation and the original text
should properly not overstep the first level. To the possible
differences here belong also the coloring and shading which poetic
eloquence seeks to give to the sense. Such coloring and shading are
not objective, and must be evoked by each hearer or reader according
to the hints of the poet or the speaker. Without some affinity in
human conceptions art would certainly be impossible; but it can never
be exactly determined how far the intentions of a poet are realized.

In what follows there will be no further discussion of conceptions and
experiences; they have been mentioned here only to ensure that the
conception aroused in the hearer by a word shall not be confused with
its sense or its referent.

To make short and exact expressions possible, let the following
phraseology be established:

A proper name (word, sign, sign combination, expression)
\emph{expresses} its sense, \emph{refers to} or \emph{designates} its
referent. By means of a sign we express its sense and designate its
referent.

Idealists or skeptics will perhaps long since have objected: ``You
talk, without further ado, of the moon as an object; but how do you
know that the name `the moon' has any referent? How do you know that
anything whatsoever has a referent?'' I reply that when we say ``the
moon,'' we do not intend to speak of our conception of the moon, nor
are we satisfied with the sense alone, but we presuppose a referent.
To assume that in the sentence ``The moon is smaller than the earth''
the conception of the moon is in question, would be flatly to
misunderstand the sense. If this is what the speaker wanted, he would
use the phrase ``my conception of the moon.'' Now we can of course be
\marginnote{32}
mistaken in the presupposition, and such mistakes have indeed
occurred. But the question whether the presupposition is perhaps
always mistaken need not be answered here; in order to justify mention
of the referent of a sign it is enough, at first, to point out our
intention in speaking or thinking. (We must then add the reservation:
provided such a referent exists.)

So far we have considered the sense and referents only of such
expressions, words, or signs as we have called proper names. We now
inquire concerning the sense and referent of an entire declarative
sentence. Such a sentence contains a thought.\footnote[5]{By a thought
  I understand not the subjective performance of thinking but its
  objective content, which is capable of being the common property of
  several thinkers.} Is this thought, now, to be regarded as its sense
or its referent? Let us assume for the time being that the sentence
has a referent! If we now replace one word of the sentence by another
having the same referent, but a different sense, this can have no
influence upon the referent of the sentence. Yet we can see that in
such a case the thought changes; since, e.g., the thought of the
sentence ``The morning star is a body illuminated by the sun'' differs
from that of the sentence ``The evening star is a body illuminated by
the sun.'' Anybody who did not know that the evening star is the
morning star might hold the one thought to be true, the other false.
The thought, accordingly, cannot be the referent of the sentence, but
must rather be considered as the sense. What is the position now with
regard to the referent? Have we a right even to inquire about it? Is
it possible that a sentence as a whole has only a sense, but no
referent? At any rate, one might expect that such sentences occur,
just as there are parts of sentences having sense but no referent. And
sentences which contain proper names without referents will be of this
kind. The sentence ``Odysseus was set ashore at Ithaca while sound
asleep'' obviously has a sense. But since it is doubtful whether the
name ``Odysseus,'' occurring therein, has a referent, it is also
doubtful whether the whole sentence has one. Yet it is certain,
nevertheless, that anyone who seriously took the sentence to be true
or false would ascribe to the name ``Odysseus'' a referent, not merely
\marginnote{33}
a sense; for it is the referent of the name which is held to be or not
to be characterized by the predicate. Whoever does not consider the
referent to exist, can neither apply nor withhold the predicate. But
in that case it would be superfluous to advance to the referent of the
name; one could be satisfied with the sense, if one wanted to go no
further than the thought. If it were a question only of the sense of
the sentence, the thought, it would be unnecessary to bother with the
referent of a part of the sentence; only the sense, not the referent,
of the part is relevant to the sense of the whole sentence. The fact
that we concern ourselves at all about the referent of a part of the
sentence indicates that we generally recognize and expect a referent
for the sentence itself. The thought loses value for us as soon as we
recognize that the referent of one of its parts is missing. We are
therefore justified in not being satisfied with the sense of a
sentence, and in inquiring also as to its referent. But now why do we
want every proper name to have not only a sense, but also a referent?
Why is the thought not enough for us? Because, and to the extend that,
we are concerned with its truth value. This is not always the case. In
hearing an epic poem, for instance, apart from the euphony of the
language we are interested only in the sense of the sentences and the
images and feelings thereby aroused. The question of truth would cause
us to abandon aesthetic delight for an attitude of scientific
investigation. Hence it is a matter of indifference to use whether the
name ``Odysseus,'' for instance, has a referent, so long as we accept
the poem as a work of art.\footnote[6]{It would be desirable to have a
  special term for signs having only sense. If we name them, say,
  representations, the words of the actors on the stage would be
  representations; indeed the actor himself would be a
  representation.} It is the striving for truth that drives us always
to advance from the sense to the referent.

We have seen that the referent of a sentence may always be sought,
whenever the referents of its components are involved; and that this
is the case when and only when we are inquiring after the truth value.

\marginnote{34}
We are therefore driven into accepting the \emph{truth value} of a
sentence as its referent. By the truth value of a sentence I
understand the circumstance that it is true or false. There area no
further truth values. For brevity I call the one the true, and the
other the false. Every declarative sentence concerned with the
referents of its words is therefore to be regarded as a proper name,
and its referent, if it exists, is either the true or the false. These
two objects are recognized, if only implicitly, by everybody who
judges something to be true---and so even by a skeptic. The designation
of the truth values as objects may appear to be an arbitrary fancy or
perhaps a mere play upon words, from which no profound consequences
could be drawn. What I mean by an object can be more exactly discussed
only in connection with concept and relation. I will reserve this for
another article.\footnoteAlph[3]{See his ``Ueber Begriff und
  Gegenstand'' in {\it Vierteljahrsschrift f\"ur wissenschaftliche
    Philosophie} (XVI [1892], 192-205).} But so much should already be
clear, that in every judgment,\footnote[7]{A judgment, for me, is not
  the mere comprehension of a thought, but the recognition of its
  truth.} no matter how trivial, the step from the level of thoughts
to the level of referents (the objective) has already been taken.

One might be tempted to regard the relation of the thought to the true
not as that of the sense to the referent, but rather as that of
subject to predicate. One can, indeed, say: ``The thought, that 5 is a
prime number, is true.'' But closer examination shows that nothing
more has been said than in the simple sentence ``5 is a prime
number.'' The truth claim arises in each case from the form of the
declarative sentence, and when the latter lacks its usual force, e.g.,
in the mouth of an actor upon the stage, even the sentence ``The
thought that 5 is a prime number is true'' contains only a thought,
and indeed the same thought as the simple ``5 is a prime number.'' It
follows that the relation of the thought to the true may not be
compared with that of subject to predicate. \marginnote{35} Subject
and predicate (understood in the logical sense) are indeed elements of
thought; they stand on the same level for knowledge. By combining
subject and predicate, one reaches only a thought, never passes from a
sense to its referent, never from a thought to its truth value. One
moves at the same level but never advances from one level to the next.
A truth value cannot be a part of a thought, any more than say the sun
can, for it is not a sense but an object.

If our supposition that the referent of a sentence is its truth value
is correct, the latter must remain unchanged when a part of the
sentence is replaced by an expression having the same referent. And
this is in fact the case. Leibniz explains: {\it ``Eadem sunt, quae
  sibi mutuo substitui possunt, salva veritat.''} What else but the
truth value could be found, that belongs quite generally to every
sentence concerned with the referents of its components and remains
unchanged by substitutions of the kind in question?

If now the truth value of a sentence is its referent, then on the one
hand all true sentences have the same referent and so, on the other
hand, do all false sentences. From this we see that in the referent of
the sentence all that is specific is obliterated. We can never be
concerned only with the referent of a sentence; but again the mere
thought alone yields no knowledge, but only the thought together with
its referent, i.e., its truth value. Judgments can be regarded as
advances from a thought to a truth value. Naturally this cannot be a
definition. Judgment is something quite peculiar and incomparable. One
might also say that judgments are distinctions of parts within truth
values. Such distinction occurs by a return to the thought. To every
sense belonging to a truth value there would correspond its own manner
of analysis. However, I have here used the word ``part'' in a special
sense. I have in fact transferred the relation between the parts and
the whole of the sentence to its referent, by calling the referent of
\marginnote{36}
a word part of the referent of the sentence, if the word itself is a
part of the sentence. This way of speaking can certainly be attacked,
because in the case of a referent the whole and one part do not
suffice to determine the remainder, and because the word part is
already used in another sense of bodies. A special term would need to
be invented.

The supposition that the truth value of a sentence is its referent
shall now be put to further test. We have found that the truth value
of a sentence remains unchanged when an expression is replaced by
another having the same referent: But we have not yet considered the
case in which the expression to be replaces is itself a sentence. Now
if our view is correct, the truth value of a sentence containing
another as part must remain unchanged when the part is replaced by
another sentence having the same truth value. Exceptions are to be
expected when the whole sentence or its part is direct or indirect
quotation; for in such cases, as we have seen, the words do not have
their customary referents. In direct quotation, a sentence designates
another sentence, and in indirect quotation a thought.

We are thus led to consider subordinate sentences or clauses. These
occur as parts of a sentence structure, which is, from the logical
standpoint, likewise a sentence. But here we meet the question whether
it is also true of the subordinate sentence that its referent is a
truth value. Of indirect quotation we already know the opposite.
Grammarians view subordinate clauses as representatives of parts of
sentences and divide them accordingly into noun clauses, adjective
clauses, adverbial clauses. This might generate the supposition that
the referent of a subordinate clause was not a truth value but rather
of the same kind as the referent of a noun or adjective or adverb---in
short, of a part of a sentence, whose sense was not a thought but only
a part of a thought. Only a more thorough investigation can clarify
the issue. In so doing, we shall not follow the grammatical categories
strictly, but rather group together what is logically of the same
kind. Let us first search for cases in which the sense of the
subordinate clause, as we have just supposed, is not an independent
thought.

\marginnote{37}
The case of an abstract\footnoteAlph[4]{A literal translation of
  Frege's ``abstracten Nenns\"atzen'' whose meaning eludes me.} noun
clause, introduced by ``that,'' includes the case of indirect
quotation, in which we have seen the words to have their indirect
referents coinciding with what is customarily their sense. In this
case, then, the subordinate clause has for its referent a thought, not
a truth value; as sense not a thought, but the sense of the words
``the thought, that ...,'' which is only a part of the thought of the
entire complex sentence. This happens after ``say,'' ``here,'' ``be of
the opinion,'' ``be convinced,'' ``conclude,'' and similar
words.\footnote[8]{In ``A lied in saying he had seen $B$,'' the
  subordinate clause designates a thought which is said (1) to have
  been asserted by $A$ (2) while A was convinced of its falsity.}
Otherwise, and indeed somewhat complicated, is the situation after
words like ``perceive,'' ``know,'' ``fancy,'' which are to be
considered later.

That in the cases of the first kind the referent of the subordinate
clause is in fact the thought can also be recognized by seeing that it
is indifferent to the truth of the whole whether the subordinate
clause is true or false. Let us compare, for instance, the two
sentences ``Copernicus believed that the planetary orbits are
circles'' and ``Copernicus believed that the apparent motion of the
sun is produced by the real motion of the earth.'' One subordinate
clause can be substituted for the other without harm to the truth. The
main clause and the subordinate clause together have as their sense
only a single thought, and the truth of the whole includes neither the
truth nor the untruth of the subordinate clause. In such cases it is
not permissible to replace one expression in the subordinate clause by
another having the same customary referent, but only by one having the
same indirect referent, i.e., the same customary sense. If somebody
were to conclude: The referent of a sentence is not its truth value,
``For then it could always be replaced by another sentence of the same
truth value,'' he would prove too much; one might just as well claim
that the referent of ``morning star'' is not Venus, since one may not
always say ``Venus'' in place of ``morning star.'' One has the right
to conclude only that the referent of a sentence is \emph{not always}
its truth value, and that ``morning star'' \marginnote{38} does not
always refer to the planet Venus, namely when the word has its
indirect referent. An exception of such a kind occurs in the
subordinate clause just considered whose referents are thoughts.

If one says ``It seems that...'' one means ``It seems to me that...''
or ``I think that...'' We therefore have the same case again. The
situation is similar in the case of expressions such as ``to be
pleased,'' ``to regret,'' ``to approve,'' ``to blame,'' ``to hope,''
``to fear.'' If, toward the end of the battle of
Waterloo,\footnoteAlph[5]{Frege uses the Prussian name for the
  battle---``Belle Alliance.''} Wellington was glad that the Prussians
were coming, the basis for his joy was a conviction. Had he been
deceived, he would have been no less pleased so long as his illusion
lasted; and before he became so convinced he could not have been
pleased that the Prussians were coming---even though in fact they might
have been already approaching.

Just as conviction or a belief is the ground of a feeling, it can, as
in inference, also be the ground of a conviction. In the sentence:
``Columbus inferred from the roundness of the earth that he could
reach India by traveling towards the west,'' we have as referents of
the parts two thoughts, that the earth is round, and that Columbus by
traveling to the west could reach India. All that is relevant here is
that Columbus was convinced of both, and that the one conviction was a
ground for the other. Whether the earth is really round, and whether
Columbus could really reach India by traveling to the west are
immaterial to the truth of our sentence; but it is not immaterial
whether we replace ``the earth'' by ``the planet which is accompanied
by a moon whose diameter is greater than the fourth part of its own.''
Here also we have the indirect referents of the words.

Adverbial clauses of purpose beginning with ``in order to'' also
belong here; for obviously the purpose is a thought therefore:
indirect referents for the words subjunctive mood.

A subordinate clause with ``that'' after ``command,'' ``ask,''
``forbid,'' would appear in direct speech as an imperative. Such a
clause has no referent but only a sense. A command, a request, are
indeed not thoughts, yet they stand on the same level as thoughts.
Hence in subordinate \marginnote{39} clauses depending upon
``command,'' ``ask,'' etc., words have their indirect referents. The
referent of such a clause is therefore not a truth value but a
command, a request, and so forth.

The case is similar for the dependent question in phrases such as
``doubt whether,'' ``not to know what.'' It is easy to see that here
also the words are to be taken to have their indirect referents.
Dependent clauses expressing questions and beginning with ``who,''
``what,'' ``where,'' ``when,'' ``how,'' ``by what means,'' etc., seem
to at times to approximate very closely to adverbial clauses in which
words have their customary referents. These cases are distinguished
linguistically by the mood of the verb. In the case of the
subjunctive, we have a dependent question and indirect reference of
the words, so that a proper name cannot in general be replaced by
another name of the same object.

In the cases so far considered the words of the subordinate clauses
had their indirect referents, and this made it clear that the referent
of the subordinate clause itself was indirect, i.e., not a truth value
but a thought, a command, a request, a question. The subordinate
clause could be regarded as a noun, indeed one could say: as a proper
name of that thought, that command, etc., which it represented in the
context of the sentence structure.

We now come to other subordinate clauses, in which the words do have
their customary referents without however a thought occurring as sense
and a truth value as referent. How is this possible is best made clear
by examples.

\begin{quote}
  He who discovered the elliptic form of the planetary orbits died in misery.
\end{quote}

If the sense of the subordinate clause were here a thought, it would
have to be possible to express it also in a separate sentence. But
this does not work, because the grammatical subject ``he'' has no
independent sense and only mediates the relations with the consequent
clause ``died in misery.'' For this reason the sense of the
subordinate clause is not a complete thought, and its referent is
Kepler, not a truth value. One might object that the sense of the
whole does contain a thought as part, namely, that there was somebody
who first discovered the elliptic form of the planetary orbits; for
\marginnote{40}
whoever takes the whole to be true cannot deny this part. This is
undoubtedly so but only because otherwise the subordinate clause ``he
who discovered the elliptic form of the planetary orbits'' would have
no referent. If anything is asserted there is always an obvious
presupposition that the simple or compound proper names used have
referents. If one therefore asserts ``Kepler died in misery,'' there
is a presupposition that the name ``Kepler'' designates something; but
it does not follow that the sense of the sentence ``Kepler died in
misery'' contains the thought that the name ``Kepler'' designates
something. If this were the case the negation would have to run not

\begin{quote}
  Kelper did not die in misery
\end{quote}

\noindent but

\begin{quote}
  Kepler did not die in misery, or the name ``Kepler'' has no referent.
\end{quote}

\noindent That the name ``Kepler'' designates something is just as much a
presupposition for the assertion

\begin{quote}
  Kepler died in misery
\end{quote}

\noindent as for the contrary assertion. Now languages have the fault
of containing expressions which fail to designate an object (although
their grammatical form seems to qualify them for that purpose) because
the truth of some sentences is a prerequisite. Thus it depends on the
truth of the sentence:

\begin{quote}
  There was someone who discovered the elliptic form of the planetary
  orbits
\end{quote}

\noindent whether the subordinate clause

\begin{quote}
  He who discovered the elliptic form of the planetary orbits
\end{quote}

\noindent really designates an object or only seems to do so while
having in fact no referent. And thus it may appear as if our
subordinate clause contains as a part of its sense the thought that
there was somebody who discovered the elliptic form of the planetary
orbits. If this were right the negation would run:

\begin{quote}
  Either he who discovered the elliptic form of the planetary orbits
  did not die in misery or there was nobody who discovered the
  elliptic form of the planetary orbits.
\end{quote}

\noindent \marginnote{41} This arises from an incompleteness of
language, from which even the symbolic language of mathematical
analysis is not altogether free; even there combinations of symbols
can occur which appear to refer to something having (at any rate so
far) no referent, e.g., divergent infinite series. This can be
avoided, e.g., by means of the special stipulation that divergent
infinite series shall refer to the
% is this a zero or an `o'?
number 0. A logically complete language ({\it Begriffsschrift}) should
satisfy the conditions, that every expression grammatically well
constructed as proper name out of signs already introduced shall in
fact designate an object, and that no new sign shall be introduced as
a proper name without having a referent assured. The logic books
contain warnings against logical mistakes arising from ambiguity of
expressions. I regard as no less pertinent a warning against apparent
proper names having no referents. The history of mathematics supplies
errors which have arisen in this way. This lends itself to demagogic
abuse as easily as ambiguity---perhaps more easily. ``The will of the
people'' can serve as an example; for it is easy to establish that
there is at any rate no generally accepted referent for this
expression. It is therefore by no means unimportant to eliminate the
source of these mistakes, at least in science, once and for all. Then
such objections as the one discussed above would become impossible,
because it could never depend upon the truth of a thought whether a
proper name had a referent.

With the consideration of these noun clauses may be coupled that of
types of adjective and adverbial clauses which are logically closely
related to them.

Adjective clauses also serve to construct compound proper names even
if, unlike noun clauses, they are not sufficient by themselves for
this purpose. These adjective clauses are to be regarded as equivalent
to adjectives. Instead of ``the square root of 4 which is smaller than
0,'' one can say ``the negative square root of 4.'' We have here the
case of a compound proper name constructed from the predicate
expression with the help of the singular definite article. This is at
any rate permissible if the predicate applies to one and only one
single object.\footnote[9]{In accordance with what was said above, an
  expression of the kind in question must actually always be assured
  of a referent, by means of a special stipulation, e.g., by the
  convention that 0 shall count as its referent, when the predicate
  applies to no object or to more than one.}

\marginnote{42}
Predicate expressions can be so constructed that characteristics are
given by adjective clauses as, in our example, by the clause ``which
is smaller than 0.'' It is evident that such an adjective clause
cannot have a thought as sense or a truth value as referent, any more
than the noun clause could. Its sense, which can also be expressed in
many cases by a single adjective, is only a part of a thought. Here,
as in the case of the noun clause, there is no independent subject and
therefore no possibility of reproducing the sense of the subordinate
clause in an independent sentence.

Places, instants, stretches of time, are logically considered,
objects; hence the linguistic designation of a definite place, a
definite instant, or a stretch of time is to be regarded as a proper
name. Now adverbial clauses of place and time can be used for the
construction of such a proper name in a manner similar to that which
we have seen in the case of noun adjective clauses. In the same way,
predicate expressions containing reference to places, etc., can be
constructed. it is to be noted here also that the sense of these
subordinate clauses cannot be reproduced in an independent sentence,
since an essential component, namely the determination of place or
time, is missing and is only indicated by a relative pronoun or a
conjunction.\footnote[10]{In the case of these sentences, various
  interpretations are easily possible. The sense of the sentence,
  ``After Schleswig-Holstein was separated from Denmark, Prussia, and
  Austria quarreled'' can also be rendered in the form ``After the
  separation of Schleswig-Holstein from Denmark, Prussia and Austria
  quarreled.'' In this version, it is surely sufficiently clear that
  the sense is not to be taken as having as a part the thought that
  Schleswig-Holstein was once separated from Denmark, but that this is
  the necessary presupposition in order for the expression ``after the
  separation of Schleswig-Holstein from Denmark'' to have any referent
  at all. To be sure, our sentence can also be interpreted as saying
  Schleswig-Holstein was once separated from Denmark. We then have a
  case which is to be considered later. In order to understand the
  difference more clearly, let us project ourselves into the mind of a
  Chinese who, having little knowledge of European history, believes
  it to be false that Schleswig-Holstein was ever separated from
  Denmark. He will take our sentence, in the first version, to be
  neither true nor false but will deny it to have any referent, on the
  ground of absence of referent for its subordinate clause. This clause
  would only apparently determine a time. If he interpreted our
  sentence in the second way, however, he would find a thought
  expressed in it, which he would take to be false, beside a part
  which would be without reference for him.}

\marginnote{43}
In conditional clauses, also, there may usually be recognized to occur
an indefinite indicator, having a similar correlate in the dependent
clauses. (We have already seen this occur in noun, adjective, and
adverbial clauses.) Insofar as each indicator refers to the other,
both clauses together form a connected whole, which as a rule
expresses only a single thought. In the sentence

\begin{quote}
  If a number is less than 1 and greater than 0, its square is less
  than 1 and greater than 0
\end{quote}

\noindent the component in question is ``a number'' in the conditional
clause and ``its'' in the dependent clause. It is by means of this
very indefiniteness that the sense acquires the generality expected of
a law. It is this which is responsible for the fact that the
antecedent clause alone has no complete thought as its sense and in
combination with the consequent clause expresses one and only one
thought, whose parts are no longer thoughts. It is, in general,
incorrect to say that in the hypothetical judgment two judgments are
put in reciprocal relationship. If this or something similar is said,
the word ``judgment'' is used in the same sense as I have connected
with the word ``thought,'' so that I would use the formulation: ``A
hypothetical thought establishes a reciprocal relationship between two
thoughts.'' This could be true only if an indefinite indicator is
absent;\footnote[11]{At times an explicit linguistic indication is
  missing and must be read off from the entire context.} but in such a
case there would also be no generality.

If an instant of time is to be indefinitely indicated in both
conditional and dependent clauses, this is often achieved merely by
using the present tense of the verb, which in such a case however does
not indicate the temporal present. This grammatical form is then the
indefinite indicator in the main subordinate clauses. An example of
\marginnote{44}
this is: ``When the sun is in the tropic of cancer, the longest day
in the northern hemisphere occurs.'' Here, also, it is impossible to
express the sense of the subordinate clause in a full sentence,
because this sense is not a complete thought. If we say: ``The sun is
in the tropic of cancer,'' this would refer to our present time and
thereby change the sense. Just as little is the sense of the main
clause a thought only the whole, composed of main and subordinate
clauses, is such. It may be added that several common components in
the antecedent and consequent clauses may be indefinitely indicated.

It is clear that noun clauses with ``who'' or ``what'' and adverbial
clauses with ``where,'' ``when,'' ``wherever,'' ``whenever'' are often
to be interpreted as having the sense of conditional clauses, e.g.,
``who touches pitch, defiles himself.''

Adjective clauses can also take the place of conditional clauses. Thus
the sense of the sentence previously used can be given in the form
``The square of a number which is less than 1 and greater than 0 is
less than 1 and greater than 0.''

The situation is quite different if the common component of the two
clauses is designated by a proper name. In the sentence:

\begin{quote}
  Napoleon, who recognized the danger to his right flank, himself led
  his guards against the enemy position
\end{quote}

\noindent two thoughts are expressed:

\begin{enumerate}
\item Napoleon recognized the danger to his right flank
\item Napoleon himself led his guards against the enemy position.
\end{enumerate}

\noindent When and where this happened is to be fixed only by the
context, but is nevertheless to be taken as definitely determined
thereby. If the entire sentence is uttered as an assertion, we thereby
simultaneously assert both component sentences. If one of the parts is
false, the whole is false. Here we have the case that the subordinate
clause by itself has a complete thought as sense (if we complete it by
indication of place and time). The referent of the subordinate clause
is accordingly a truth value. We can therefore expect that it may be
replaced, without harm to the truth value of the whole, by a sentence
having the \marginnote{45} same truth value. This is indeed the case;
but it is to be noticed that for purely grammatical reasons, its
subject must be ``Napoleon,'' for only then can it be brought into the
form of an adjective clause belonging to ``Napoleon.'' But if the
demand that it be expressed in this form be waived, and the connection
be shown by ``and,'' this restriction appears.

Subsidiary clauses beginning with ``although'' also express complete
thoughts. This conjunction actually has no sense and does not change
the sense of the clause but only illuminates it in a peculiar
fashion.\footnote[12]{Similarly in the case of ``but,'' ``yet.''} We
could indeed replace the conditional clause without harm to the truth
of the whole by another of the same truth value; but the light in
which the clause is placed by the conjunction might then easily appear
unsuitable, as if a song with a sad subject were to be sung in a
lively fashion.

In the last cases the truth of the whole included the truth of the
component clauses. The case is different if a conditional clause
expressed a complete thought by containing, in place of an indefinite
indicator, a proper name or something which is to be regarded as
equivalent. In the sentence

\begin{quote}
  If the sun has already risen, the sky is very cloudy
\end{quote}

\noindent the time is the present, that is to say, definite. And the
place is also to be thought of as definite. Here it can be said that a
relation between the truth values of conditional and dependent clauses
has been asserted, namely such that the case does not occur in which
the antecedent clause refers to the true and the consequent to the
false. Accordingly, our sentence is true when the sun has not yet
risen, whether the sky is very cloudy or not, and also when the sun
has risen and the sky is very cloudy. Since only truth values are here
in question, each component clause can be replaced by another of the
same truth value without changing the truth value of the whole. To be
sure, the light in which the subject then appears would usually be
unsuitable; the \marginnote{46} thought would easily seem distorted;
but this has nothing to do with its truth value. One must always take
care not to clash with the subsidiary thoughts, which are however not
explicitly expressed and therefore should not be reckoned in the
sense. Hence, also, no account need be taken of their truth
values.\footnote[13]{The thought of our sentence might also be
  expressed thus: ``Either the sun has not risen yet or the sky is
  very cloudy''---which shows how this kind of sentence connection is to
  be understood.}

The simple cases have now been discussed. Let us review what we have
learned!

The subordinate clause usually has for its sense not a thought, but
only a part of one, and consequently no truth value as referent. The
reason for this is either that the words in the subordinate clause
have indirect reference, so that the referent, not the sense, of the
subordinate clause is a thought; or else that, on account of the
presence of an indefinite indicator, the subordinate clause is
incomplete and expresses a thought only when combined with the main
clause. It may happen, however, that the sense of the subsidiary
clause is a complete thought, in which case it can be replaced by
another of the same truth value without harm to the truth of the
whole---provided there are no grammatical obstacles.

An examination of all the subordinate clauses which one may encounter
will soon provide some which do not fit well into these categories.
The reason, so far as I can see, is that these subordinate clauses
have no such simple sense. Almost always, it seems, we connect with
the main thoughts expressed by us subsidiary thoughts which, although
not expressed, are associated with our words, in accordance with
psychological laws, by the hearer. And since the subsidiary thought
appears to be connected with our words of its own accord, almost like
the main thought itself, we want it also to be expressed. The sense of
the sentence is thereby enriched, and it may well happen that we have
more simple thoughts than clauses. In many cases the sentence must be
understood in this way, in others it may be doubtful whether the
subsidiary thought belongs to the sense of the sentence or only
\marginnote{47}
accompanies it.\footnote[14]{This may be important for the question
  whether an assertion is a lie, or an oath a perjury} One might
perhaps find that the sentence

\begin{quote}
  Napoleon, who recognized the danger to his right flank, himself led
  his guards against the enemy position
\end{quote}

\noindent expresses not only the two thoughts show above, but also the
thought that the knowledge of the danger was the reason why he led the
guards against the enemy position. One may in fact doubt whether this
thought is merely lightly suggested or really expressed. Let the
question be considered whether our sentence be false if Napoleon's
decision had already been made before he recognized the danger. If our
sentence could be true in spite of this, the subsidiary thought should
not be understood as part of the sense. One would probably decide in
favor of this. The alternative would make for a quite complicated
situation: We would have more simple thoughts than clauses. If the
sentence

\begin{quote}
  Napoleon, recognized the danger to his right flank
\end{quote}

\noindent were now to be replaced by another having the same truth
value, e.g.,

\begin{quote}
  Napoleon was already more than 45 years old
\end{quote}

\noindent not only would our first thought be changed, but also our
third one. Hence the truth value of the latter might change---namely, if
his age was not the reason for the decision to lead the guards against
the enemy. This shows why clauses of equal truth value cannot always
be substituted for one another in such cases. The clause expresses
more through its connection with another than it does in isolation.

Let us now consider cases where this regularly happens In the
sentence:

\begin{quote}
  Bebel mistakenly supposes that the return of Alsace-Lorraine would
  appease France's desire of revenge
\end{quote}

two thoughts are expressed, which are not however shown by means of
antecedent and consequent clauses, viz.:

\begin{enumerate}[label={(\arabic*)}]
\item Bebel believes that the return of Alsace-Lorraine would appease
  France's desire fore revenge
\item \marginnote{48} the return of Alsace-Lorraine would not appease France's desire
  for revenge.
\end{enumerate}

In the expression of the first thought, the words of the subordinate
clause have their indirect referent while the same words have their
customary referents in the expression of the second thought. This
shows that the subordinate clause in our original complex sentence is
to be taken twice over, with different referents, of which one is a
thought, the other a truth value. Since the truth value is not the
whole referent of the subordinate clause, we cannot simply replace the
latter by another of equal truth value. Similar considerations apply
to expressions such as ``know,'' ``discover,'' ``it is known that.''

By means of a subordinate clause of reason and the associated main
clause we express several thoughts, which however do not correspond
separately to the original clauses. In the sentence:

``Because ice is less dense than water, it floats on water'' we have

\begin{enumerate}[label={(\arabic*)}]
\item Ice is less dense than water;
\item If anything is less dense than water, it floats on water;
\item Ice floats on water.
\end{enumerate}

The third thought, however, need not be explicitly introduced, since
it is contained in the remaining two. On the other hand, neither the
first and third nor the second and third combined would furnish the
sense of our sentence. It can now be seen that our subordinate clause

\begin{quote}
  because ice is less dense than water
\end{quote}

\noindent expresses our first thought, as well as a part of our second. This is
how it comes to pass that our subsidiary clause cannot be simply
replaced by another of equal truth value; for this would alter our
second thought and thereby easily alter its truth value.

The situation is similar in the sentence

\begin{quote}
  If iron were less dense than water, it would float on water.
\end{quote}

\marginnote{49}
Here we have the two thoughts that iron is not less dense than water,
and that something floats on water if it is less dense than water. The
subsidiary clause again expresses one thought and a part of the other.

If we interpret the sentence already considered

\begin{quote}
  After Schleswig-Holstein was separated from Denmark, Prussia and
  Austria quarreled
\end{quote}

\noindent in such a way that it expresses that Schleswig-Holstein was
once separated from Denmark, we have first this thought, and secondly
the thought that at a time, more closely determined by the subordinate
clause, Prussia and Austria quarreled. Here also the subordinate
clause expresses not only one thought but also a part of another.
Therefore it may not in general be replaced by another of the same
truth value.

It is hard to exhaust all the possibilities given by language; but I
hope to have brought to light at least the essential reasons why a
subordinate clause may not always be replaced by another of equal
truth value without harm to the truth of the whole sentence structure.
These reasons arise:

\begin{enumerate}[label={(\arabic*)}]
\item when the subordinate clause does not refer to a truth value,
  inasmuch as it expresses only part of a thought;
\item when the subordinate clause does refer to a truth value but is
  not restricted to so doing, inasmuch as its sense includes one
  thought and part of another.
\end{enumerate}

The first case arises:
\begin{enumerate}[label={(\alph*)}]
\item in indirect reference of words
\item if a part of the sentence is only an indefinite indicator
  instead of a proper name.
\end{enumerate}

In the second case, the subsidiary clause may have to be taken twice
over, viz., once in its customary reference, and the other time in
indirect reference; or the sense of a part of the subordinate clause
may likewise be a component of another thought, which, taken together
with the thought directly expressed by the subordinate clause, makes
up the sense of the whole sentence.

It follows with sufficient probability from the foregoing that the
cases where a subordinate clause is not replaceable by another of the
\marginnote{50}
same value cannot be brought in disproof of our view that a truth
value is the referent of a sentence having a thought as its sense.

Let us return to our starting point!

If we found ``$a=a$'' and ``$a=b$'' to have different cognitive
values, the explanation is that for the purpose of knowledge, the
sense of the sentence, viz., the thought expressed by it, is no less
relevant than its referent, i.e., its truth value. If now $a=b$, then
indeed the referent of ``$b$'' is the same as that of ``$a$,'' and
hence the truth value of ``$a=b$'' is the same as that of ``$a=a$.''
In spite of this, the sense of ``$b$'' may differ from that of
``$a=a$.'' In that case the two sentences do not have the same
cognitive value. If we understand by ``judgment'' the advance from
the thought to its truth value, as in the above paper, we can also say
that the judgments are different.

\end{document}

%%% Local Variables:
%%% mode: latex
%%% TeX-master: t
%%% eval: (TeX-source-correlate-mode 1)
%%% End:

%% English spellcheckers when they see German/Latin be like

%% LocalWords:  Gottlob Frege Begriffsschrift Frege's Halle eine SINN
%% LocalWords:  Beigriffsschrift der arithmetischen nachgebildete UND
%% LocalWords:  Farmelsprache des reinen Denkens Stagira faciunt und
%% LocalWords:  Bucephalus UEBER BEDEUTUNG Kelper Ueber Begriff ur
%% LocalWords:  Gegenstand Vierteljahrsschrift wissenschaftliche
%% LocalWords:  Philosophie Eadem sunt quae sibi mutuo substitui
%% LocalWords:  possunt salva veritat Schleswig Bebel
